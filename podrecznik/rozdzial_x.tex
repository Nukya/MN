\pdfminorversion=7
\documentclass[a4paper,12pt]{article}
\usepackage[utf8]{inputenc}
\usepackage[T1]{fontenc}
\usepackage{polski}
\usepackage{lmodern}
\usepackage{geometry}
\geometry{margin=2.5cm}
\usepackage{amsmath, amssymb, amsfonts}
\usepackage{mathtools}
\usepackage{bm}
\usepackage{physics}
\usepackage{siunitx}
\usepackage{float}
\usepackage{graphicx}
\usepackage{booktabs}
\usepackage{enumitem}
\usepackage{hyperref}
\usepackage{pdfpages}

% Zgodno\'{s}c siunitx z komend\k{a} \qty z pakietu physics.
\AtBeginDocument{\RenewCommandCopy\qty\SI}

% Dodatkowe deklaracje operatorów wykorzystywane w notatkach.
\DeclareMathOperator{\sgn}{sgn}
\DeclareMathOperator{\spann}{span}
\DeclareMathOperator{\diag}{diag}
\DeclareMathOperator{\argmin}{arg\,min}
\DeclareMathOperator{\argmax}{arg\,max}

% Skróty na często używane symbole macierzowe i wektorowe.
\newcommand{\mat}[1]{\bm{#1}}
\newcommand{\vect}[1]{\bm{#1}}
\newcommand{\R}{\mathbb{R}}
\newcommand{\N}{\mathbb{N}}

\title{Rozdzia\l{} X -- poprawione wstawki \LaTeX{}}
\author{}
\date{\today}

\begin{document}
\maketitle

\section*{Instrukcja u\{z}ycia}
Zawarto\'{s}\'{c} niniejszego pliku stanowi szablon do rozdzia\l{}u X, w kt\'orym mo\.{z}na umieszcza\'c fragmenty matematyczne z plik\'ow PDF znajduj\k{a}cych si\k{e} w katalogu \texttt{podrecznik}. Dodane pakiety i definicje zapewniaj\k{a} poprawne renderowanie najcz\k{e}\'{s}ciej u\.{z}ywanych wstawiek (pochodne cz\k{e}sciowe, normy, warto\'{s}ci bezwzgl\k{e}dne, wektory i macierze). W razie potrzeby mo\.{z}na skopiowa\'c odpowiednie wzory z konkretnych rozdzia\l\'ow i wklei\'c je do sekcji poni\.{z}ej.

% Aby szybko podgl\k{a}da\'c oryginalny rozdzia\l{}, odkomentuj jedynie potrzebny plik.
% \includepdf[pages=-]{1 Rozdzial.pdf}
% \includepdf[pages=-]{2 Rozdzial.pdf}
% \includepdf[pages=-]{3 Rozdzial.pdf}
% \includepdf[pages=-]{4 Rozdzial.pdf}
% \includepdf[pages=-]{5 Rozdzial.pdf}
% \includepdf[pages=-]{6 Rozdzial.pdf}
% \includepdf[pages=-]{7 Rozdzial.pdf}
% \includepdf[pages=-]{8 Rozdzial.pdf}
% \includepdf[pages=-]{9 Rozdzial.pdf}

\section{Przyk\l{}adowe poprawione wstawki}
\begin{itemize}[leftmargin=*]
    \item Pochodna cz\k{e}\'{s}ciowa z oznaczeniem kierunkowym:
    \begin{equation}
        \pdv{f}{x_i}(\vect{x}_0) = \lim_{h\to 0} \frac{f(\vect{x}_0 + h\,\vect{e}_i) - f(\vect{x}_0)}{h}.
    \end{equation}
    \item Norma euklidesowa macierzy i przekszta\l{}cenie liniowe:
    \begin{equation}
        \norm{A} = \sup_{\vect{x}\in\R^n \setminus \{\vect{0}\}} \frac{\norm{A\vect{x}}_2}{\norm{\vect{x}}_2}, \qquad A \in \R^{n\times n}.
    \end{equation}
    \item R\'ownanie rozwoju Taylora z reszt\k{a} w postaci Lagrange'a:
    \begin{equation}
        f(x) = \sum_{k=0}^{m} \frac{f^{(k)}(x_0)}{k!}(x-x_0)^k + \frac{f^{(m+1)}(\xi)}{(m+1)!}(x-x_0)^{m+1}, \quad \xi \in (x_0, x).
    \end{equation}
    \item Uk\l{}ad r\'owna\'n liniowych w zapisie blokowym:
    \begin{equation}
        \begin{bmatrix}
            A & B \\
            C & D
        \end{bmatrix}
        \begin{bmatrix}
            \vect{x} \\
            \vect{y}
        \end{bmatrix} =
        \begin{bmatrix}
            \vect{f} \\
            \vect{g}
        \end{bmatrix},
        \qquad A \in \R^{n\times n},\; B \in \R^{n\times m}.
    \end{equation}
    \item Notacja na iteracyjny krok metody numerycznej:
    \begin{equation}
        \vect{x}^{(k+1)} = \vect{x}^{(k)} - J_f(\vect{x}^{(k)})^{-1} f(\vect{x}^{(k)}),
    \end{equation}
    gdzie $J_f$ oznacza macierz Jacobiego, a $k$ jest numerem iteracji.
    \item Prosta tabela wynik\'ow z legend\k{a} dla symboli:
    \begin{table}[H]
        \centering
        \caption{Przyk\l{}ad formatowania danych}\label{tab:wzor}
        \begin{tabular}{@{}lll@{}}
            \toprule
            Wielko\'{s}\'{c} & Symbol & Warto\'{s}\'{c} \\
            \midrule
            B\l{}\k{a}d wzgl\k{e}dny & $\varepsilon_r$ & $1.2\times 10^{-3}$ \\
            Liczba iteracji & $k$ & $12$ \\
            Norma residuum & $\norm{r_k}_2$ & $4.5\times 10^{-6}$ \\
            \bottomrule
        \end{tabular}
    \end{table}
\end{itemize}

\section{Miejsce na wstawki z konkretnych rozdzia\l\'ow}
% Wklej tutaj fragmenty wzor\'ow przeniesione z PDF-\'ow, korzystaj\k{a}c z powy\.{z}szego
% preambu\l{}y. Na przyk\l{}ad:
% \begin{align}
%     y'(x) &= f(x, y(x)), \\
%     y(x_0) &= y_0.
% \end{align}

\end{document}
